\section{Template Section}
\label{sec:Template}       % use below command to reference this section
                           % \cref{sec:Template}

% =========================================================
%                        SECTIONS 
% =========================================================
\subsection{This is a subsection}
    \subsubsection{This is a subsubsection}
    This is text.

% =========================================================
%                  GENERAL FORMATTING
% =========================================================
\subsection{General formatting}
Firstly, the document uses the font mlmodern, using no indent for new paragraphs and commonly uses the color \textcolor{Tue-red}{Tue-red}.
The \texttt{hyperref} package is responsible for highlighting and formatting references like figures and tables. For example \cref{table: style 1} or \cref{fig: three images}. It also works for citations \cite{texbook}. Note how figure numbers are numbered according to the format \texttt{<chapter number>.<figure number>}.\\

% =========================================================
%                       Bulleted Lists 
% =========================================================
\subsection{Bulleted Lists}
Bullet lists are also changed globally, for a maximum of 3 levels:

\begin{itemize}
    \item Item 1
    \begin{itemize}
        \item subitem 1
        \begin{itemize}
            \item subsubitem 1
            \item subsubitem 2
        \end{itemize}
    \end{itemize}
\end{itemize}

Similarly numbered lists are also changed document wide:

\begin{enumerate}
    \item Item 1
    \item Item 2
    \begin{enumerate}
        \item subitem 1
        \begin{enumerate}
            \item subsubitem 1
        \end{enumerate}
    \end{enumerate}
\end{enumerate}


% =========================================================
%                         TABLES 
% =========================================================
\subsection{Tables}
The following table, \cref{table: style 1}, shows a possible format for tables in this document. Alternatively, one can also use the black and white version of this, shown in \cref{table: style 2}. Note that caption labels are in the format \textbf{\textcolor{Tue-red}{Table x.y:} }

% Table 1 (style 1) ---------------------------
\begin{table}[ht]
    \rowcolors{2}{Tue-red!10}{white}
    \centering
    \caption{A table without vertical lines.}
    \begin{tabular}[t]{ccccc}
        \toprule
        \color{Tue-red}\textbf{Column 1}&\color{Tue-red}\textbf{Column 2}&\color{Tue-red}\textbf{Column 3}&\color{Tue-red}\textbf{Column 4}&\color{Tue-red}\textbf{Column 5}\\
        \midrule
        Entry 1&1&2&3&4\\
        Entry 2&1&2&3&4\\
        Entry 3&1&2&3&4\\
        Entry 4&1&2&3&4\\
        \bottomrule
    \end{tabular}
    \label{table: style 1}
\end{table}

% Table 2 (style 2) ---------------------------
\begin{table}[ht]
    \rowcolors{2}{gray!10}{white}
    \centering
    \caption{A table without vertical lines.}
    \begin{tabular}[t]{ccccc}
        \toprule
        \textbf{Column 1}&\textbf{Column 2}&\textbf{Column 3}&\textbf{Column 4}&\textbf{Column 5}\\
        \midrule
        Entry 1&1&2&3&4\\
        Entry 2&1&2&3&4\\
        Entry 3&1&2&3&4\\
        Entry 4&1&2&3&4\\
        \bottomrule
    \end{tabular}
    \label{table: style 2}
\end{table}


% =========================================================
%                        FIGURES 
% =========================================================
\subsection{Figures}
For normal, single image figures, the standard \texttt{\textbackslash begin\{figure\}} environment can be used. 

% Single Figure -------------------------------------------
\begin{figure}[h]
    \centering
    \includegraphics[width=8cm]{example-image-a}
    \label{fig: style 0 image a}
    \caption{Single image}
\end{figure}


% Multi-Figure (linked captions) --------------------------
For multi-image figures, one could use either the \texttt{\textbackslash begin\{subfigure\}} environment to get a main caption with 3 subcaptions like \cref{fig: three images}:

\begin{figure}[h]
     \centering
     \begin{subfigure}[b]{0.3\textwidth}
         \centering
         \includegraphics[width=\textwidth]{example-image-a}
         \caption{image a}
         \label{fig: style 1 image a}
     \end{subfigure}
     \hfill
     \begin{subfigure}[b]{0.3\textwidth}
         \centering
         \includegraphics[width=\textwidth]{example-image-b}
         \caption{image b}
         \label{fig: style 1 image b}
     \end{subfigure}
     \hfill
     \begin{subfigure}[b]{0.3\textwidth}
         \centering
         \includegraphics[width=\textwidth]{example-image-c}
         \caption{image c}
         \label{fig: style 1 image c}
     \end{subfigure}
        \caption{Three images}
        \label{fig: three images}
\end{figure}


% Multi-Figure (independant captions) ---------------------------
 Or one could use the \texttt{\textbackslash begin\{minipage\}} environment to get 3 independent captions like \cref{fig: style 2 image a} - \ref{fig: style 2 image c}

\begin{figure}[h]
\centering
\begin{minipage}{0.3\textwidth}
  \centering
  \includegraphics[width=\textwidth]{example-image-a}
  \captionof{figure}{image a}
  \label{fig: style 2 image a}
\end{minipage}
\hfill
\begin{minipage}{0.3\textwidth}
  \centering
  \includegraphics[width=\textwidth]{example-image-b}
  \captionof{figure}{image b}
  \label{fig: style 2 image b}
\end{minipage}
\hfill
\begin{minipage}{0.3\textwidth}
  \centering
  \includegraphics[width=\textwidth]{example-image-c}
  \captionof{figure}{image c}
  \label{fig: style 2 image c}
\end{minipage}
\end{figure}


% =========================================================
%                      REFERENCES (citations)
% =========================================================
\subsection{Citations}
Here is some info that I got from Donald E. Knuth \cite{texbook}. \\
The citations will be displayed in the last chapter of the pdf (References). They are defined in the References.bib document in the /General folder. \\
\\
If you didn't fully understand everything, this is a reference to \cref{sec:Template}, try clicking it in the .pdf. 


